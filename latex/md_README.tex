\href{https://ci.appveyor.com/project/dpfoose/vespucci}{\tt } \href{https://travis-ci.org/VespucciProject/Vespucci}{\tt $<$img src=\char`\"{}https\+://travis-\/ci.\+org/\+Vespucci\+Project/\+Vespucci.\+svg?branch=master\char`\"{} alt=\char`\"{}\char`\"{}Travis-\/\+CI Build Status\char`\"{}\char`\"{}/$>$} \href{https://scan.coverity.com/projects/vespucciproject-vespucci}{\tt $<$img src=\char`\"{}https\+://scan.\+coverity.\+com/projects/9309/badge.\+svg\char`\"{} alt=\char`\"{}\char`\"{}Coverity Scan Build Status\char`\"{}\char`\"{}/$>$}

\hyperlink{namespace_vespucci}{Vespucci} is a free/libre/open-\/source, cross-\/platform tool for spectroscopic imaging. \hyperlink{namespace_vespucci}{Vespucci} is regularly built on Windows, Mac and Linux operating systems.

\hyperlink{namespace_vespucci}{Vespucci} is distributed under the terms of the G\+NU General Public License version 3. A copy of this license is provided in L\+I\+C\+E\+N\+SE

A research article detailing \hyperlink{namespace_vespucci}{Vespucci} has been published in the \href{http://openresearchsoftware.metajnl.com/articles/10.5334/jors.91/}{\tt {\itshape Journal of Open Research Software}}

This branch is the active branch of the project and, while more stable than the branches of individual contributors, is {\bfseries probably not stable}.

After the release of the first 1.\+0.\+0 beta, there will be tags established for releases.

Bug reports should be directed to the Issues tab.

\subsection*{Binary Releases\+: }

Releases for all three supported platforms are availible in the Releases tab. The latest binaries are available at \href{https://bintray.com/vespucciproject/Vespucci_automated_builds/Vespucci_latest}{\tt Bintray} The release package includes the executable, the \hyperlink{namespace_vespucci}{Vespucci} library and runtime pre-\/requisites and the header files needed to use the library. \subsection*{Versioning\+: }

Versions up to 1.\+0.\+0 are not considered stable and changes between 0.\+n.\+0 and 0.\+n+1.0 may be major. After 1.\+0.\+0, 1.\+n+1.0 will contain new features to 1.\+n.\+0, but will have an A\+PI that is mostly compatible. n+1.0.\+0 contains major changes in the A\+PI that may break programs built from it. 1.\+n.\+x+1 contains a bug fix or minor improvement to 1.\+n.\+x

\subsection*{Compiling \hyperlink{namespace_vespucci}{Vespucci}\+: }

Compiling \hyperlink{namespace_vespucci}{Vespucci} from source is not the easiest but still possible. Look at .travis.\+yml for linux and mac builds and appveyor.\+yml for windows builds.

\hyperlink{namespace_vespucci}{Vespucci} has the following prerequisites\+:
\begin{DoxyItemize}
\item Qt 5.\+5+
\item Armadillo (with B\+L\+A\+S/\+L\+A\+P\+A\+CK and H\+D\+F5)
\item mlpack
\item yaml-\/cpp
\item quazip
\end{DoxyItemize}

Look at the scripts used by .travis.\+yml and appveyor.\+yml to figure out how to build on your platform.

\subsection*{Contributing\+: }

If you would like to contribute to \hyperlink{namespace_vespucci}{Vespucci}, please read \hyperlink{_c_o_n_t_r_i_b_u_t_i_n_g_8md_source}{C\+O\+N\+T\+R\+I\+B\+U\+T\+I\+N\+G.\+md} If you need help setting up your development environment to build \hyperlink{namespace_vespucci}{Vespucci} from source, feel free to contact us. 